\documentclass{article}

\usepackage{arxiv}

\usepackage[utf8]{inputenc} % allow utf-8 input
\usepackage[T1]{fontenc}    % use 8-bit T1 fonts
\usepackage{hyperref}       % hyperlinks
\usepackage{url}            % simple URL typesetting
\usepackage{booktabs}       % professional-quality tables
\usepackage{amsfonts}       % blackboard math symbols
\usepackage{nicefrac}       % compact symbols for 1/2, etc.
\usepackage{microtype}      % microtypography
%\usepackage{lipsum}

\title{Time Series Analysis of Chaotic Dynamical Systems with Recurrent Neural Networks}


\author{
  Lucas Wilson \\
  Undergraduate: Mathematics, Computer Science \\
  Colorado State University\\
  Fort Collins, CO 80523 \\
  \texttt{lkwilson96@gmail.com} \\
  %% \AND
  %% Coauthor \\
  %% Affiliation \\
  %% Address \\
  %% \texttt{email} \\
  %% \And
  %% Coauthor \\
  %% Affiliation \\
  %% Address \\
  %% \texttt{email} \\
  %% \And
  %% Coauthor \\
  %% Affiliation \\
  %% Address \\
  %% \texttt{email} \\
}

\begin{document}
\maketitle

\begin{abstract}
This is a paper about chaos.
\end{abstract}

%%% INTRODUCTION %%%
\section{Introduction}

This is a paper about chaos.

%%% CHAOS %%%
\section{Chaos}

In 1963, Edward N. Lorenz published an article researching a simplified system of 
ordinary differential equations modeling a convective system \cite{lorenz1963deterministic}. His 
research popularized the possibility of a system being highly sensitive to 
initial conditions, i.e., chaos theory; while he wasn't the first to discover 
this phenomenon, he is considered to be the "official discoverer of chaos 
theory" \cite{oestreicher2007history}.

When forecasting, the goal is to have accurate predictions. However, 
a chaotic system's sensitivity to initial conditions implies that a small 
perturbation as a result of error will produce incorrect solutions.
Further, given a periodic
solution, we hope that a small perturbation, from error in measurement or from 
floating-point errors in numerical calculations, doesn't affect the solution or 
at least produces a quasi-periodic solution close to the periodic one. 
(A solution or trajectory $F$ 
is quasi-periodic if and only if for all $t$ and for some $\tau$, $F(t+\tau)$ 
is arbitrarily close to $F(t)$ \cite{lorenz1963deterministic}.)

An example of this is the trajectories of the sun and planets in our solar 
system. Newton's equations can be used to create a system of equations to model 
the bodies of mass (known as the n-body problem). However, with the existence of
other bodies of mass from other solar systems within the universe, we will have 
small perturbations introduced into the system. If the system is chaotic, then 
over time, the actual trajectories will diverge from our model's 
\cite{oestreicher2007history}. The amount of time for the trajectories to differ
by a factor of 10 is known as the Lyapunov constant 
\cite{oestreicher2007history}. 

In terms of forecasting, the Lyapunov time constant can be
used to know how much time
will pass before it's very likely that the model has become incorrect since
errors introduced to the calculated solution will have diverged too far from
the actual solution. 
Researchers have used it to evaluate the effectiveness of forecasting models on
chaotic problems \cite{pathak2018model}. Predicting the solution of
a chaotic model over an interval of time of can be easy if that interval is well
below the Lyapunov constant.

One of the defining characteristics of chaos is that the error from 
perturbations grows exponentially \cite{oestreicher2007history}. At the time of 
Lorenz' paper, unstable 
error was not well understood, and this led to Lorenz' surprising discovery 
\cite{oestreicher2007history}. Lorenz reported that a single iteration in 
calculating the solution to the dynamical system took 
approximately one second \cite{lorenz1963deterministic}. In his paper, he shows 
one of his calculations having about 3000 iterations, which must have taken 
around 50 minutes. Given the amount of time it takes to perform the calculation,
it's tempting to continue the calculation from the output of the previous 
calculation. However, the story of the discover is that while the computer used 
6 digit accuracy, it output only 3 digits, so when the simulation was continued 
by Lorenz with the less accurate measurements, he found the results to be very 
different \cite{oestreicher2007history}. 

Chaos can appear in many different situations. Lorenz's paper demonstrates chaos
in a dynamical system useful for forecasting convection in the atmosphere or 
liquids \cite{lorenz1963deterministic}. "Chaos theory has a few applications for
modeling endogenous biological rhythms such as heart rate, brain functioning, 
and biological docks" \cite{oestreicher2007history}. It can also appear in 
solutions to Hamiltonian problems such as the 3 body problem (as mentioned 
before) and the double pendulum problem (as I will show later). Understanding
the predictability of chaos is very useful to many different fields.

\subsection{Lyapunov Constant}

%%% DYNAMICAL SYSTEMS %%%
\section{Dynamical Systems}

\subsection{Lorenz System}
\subsection{Double Pendulum}
\subsection{3-Body Problem}

%%% MODELS %%%
\section{Models}

\subsection{ARIMA}
\subsection{Recurrent Neural Networks}
\subsection{Echo State Networks}
\subsection{LTSM}

%%% ANALYSIS %%%
\section{Analysis}

\subsection{Method}
\subsection{Results}

%% CONCLUSION %%%
\section{Conclusion}

\bibliography{references}
\bibliographystyle{unsrt}

\end{document}

% Useful LaTeX commands

%\section{Headings: first level}
%\subsection{Headings: second level}
%\subsubsection{Headings: third level}

%\keywords{First keyword \and Second keyword \and More}

% \paragraph{Paragraph}

%% \begin{figure}
%%   \centering
%%   \fbox{\rule[-.5cm]{4cm}{4cm} \rule[-.5cm]{4cm}{0cm}}
%%   \caption{Sample figure caption.}
%%   \label{fig:fig1}
%% \end{figure}
