\documentclass{article}

\usepackage{arxiv}
\usepackage[utf8]{inputenc} % allow utf-8 input
\usepackage[T1]{fontenc}    % use 8-bit T1 fonts
\usepackage{hyperref}       % hyperlinks
\usepackage{url}            % simple URL typesetting
\usepackage{booktabs}       % professional-quality tables
\usepackage{amsfonts}       % blackboard math symbols
\usepackage{nicefrac}       % compact symbols for 1/2, etc.
\usepackage{microtype}      % microtypography
\usepackage{amsmath}
\usepackage{lipsum}
\usepackage{caption}
\usepackage{float}
\usepackage{graphicx}
\usepackage{subcaption}

% the comment 'DATA POINT' will mark where data was calculated and then inserted into
% the paper.

\newcommand{\der}[2][t]{\frac{\mathrm{d}#2}{\mathrm{d}#1}}

\title{Time Series Analysis of Chaotic Dynamical Systems with Recurrent Neural Networks}

\author{
  Lucas Wilson \\
  Undergraduate: Mathematics, Computer Science \\
  Colorado State University\\
  Fort Collins, CO 80523 \\
  \texttt{lkwilson96@gmail.com} \\
}

\begin{document}
\maketitle

\begin{abstract}
This is a paper about chaos.
\end{abstract}

%%% INTRODUCTION %%%
\section{Introduction}

This is a paper about chaos.
% TODO: mention \texttt{echonn}

%%% CHAOS %%%
\section{Chaos}

\subsection{History}

In 1963, Edward N. Lorenz published an article researching a simplified system of 
ordinary differential equations modeling a convective system \cite{lorenz1963deterministic}. His 
research popularized the possibility of a system being highly sensitive to 
initial conditions, i.e., chaos theory; while he wasn't the first to discover 
this phenomenon, he is considered to be the "official discoverer of chaos 
theory" \cite{oestreicher2007history}.

When forecasting, the goal is to have accurate predictions. However, 
a chaotic system's sensitivity to initial conditions implies that a small 
perturbation as a result of error will produce incorrect solutions.
Further, given a periodic
solution, the hope is that a small perturbation, from error in measurement or from 
floating-point errors in numerical calculations, doesn't affect the solution or 
at least produces a quasi-periodic solution close to the periodic one. 
(A solution or trajectory $F$ 
is quasi-periodic if and only if for all $t$ and for some $\tau$, $F(t+\tau)$ 
is arbitrarily close to $F(t)$ \cite{lorenz1963deterministic}.)

An example of this is the trajectories of the sun and planets in our solar 
system. Newton's equations can be used to create a system of equations to model 
the bodies of mass (known as the n-body problem). However, with the existence of
other bodies of mass from other solar systems within the universe, 
small perturbations are introduced into the system. If the system is chaotic, then 
over time, the actual trajectories will diverge from our model's 
\cite{oestreicher2007history}. 

One of the defining characteristics of chaos is that the error from 
perturbations grows exponentially \cite{oestreicher2007history}. At the time of 
Lorenz's paper, unstable 
error was not well understood, and this led to Lorenz's surprising discovery 
\cite{oestreicher2007history}. Lorenz reported that a single iteration in 
calculating the solution to the dynamical system took 
approximately one second \cite{lorenz1963deterministic}. In his paper, he shows 
one of his calculations having about 3000 iterations, which must have taken 
around 50 minutes. Given the amount of time it takes to perform the calculation,
it's tempting to continue the calculation from the output of the previous 
calculation. However, the story of the discover is that while the computer used 
6 digit accuracy, it output only 3 digits, so when the simulation was continued 
by Lorenz with the less accurate measurements, he found the results to be very 
different \cite{oestreicher2007history}. 

Chaos can appear in many different situations. Lorenz's paper demonstrates chaos
in a dynamical system useful for forecasting convection in the atmosphere or 
liquids \cite{lorenz1963deterministic}. "Chaos theory has a few applications for
modeling endogenous biological rhythms such as heart rate, brain functioning, 
and biological docks" \cite{oestreicher2007history}. It can also appear in 
solutions to Hamiltonian problems such as the 3 body problem (as mentioned 
before) and the double pendulum problem (as I will show later). Understanding
the predictability of chaos is very useful to many different fields.

\subsection{Lyapunov Constant}

In terms of forecasting, it's useful to know how chaotic a dynamical system is. 
Chaos will be measured by the exponential rate at which nearby trajectories 
diverge. Constants describing the divergence are known as the Lyapunov 
Characteristic Exponents (LCEs) \cite{sandri1996numerical}. There is an LCE for 
each dimension, and since they all represent exponential growth. The error 
growing at the rate of the largest LCE will dominate over the others.

% TODO: talk about attractors

Given a small perturbation $\epsilon$ to a dynamical system in the direction of 
the largest LCE, we can approximate the growth of the perturbation by the 
equation $\epsilon(\Delta t) = \epsilon_0 e^{\lambda \Delta t}$ where $\lambda$ 
is the largest LCE
\cite{bezruchko2010extracting}. For an LCE of zero, then the error doesn't grow
exponentially, and thus, there is no chaos. Error created
from smaller LCEs is negligible. The approximate time scale where 
perturbations become large is going to be proportional to $1 / \lambda$. This is
known as the Lyapunov time \cite{bezruchko2010extracting}. For our solar
system, the Lyapunov time is 10,000,000 years \cite{oestreicher2007history}.
Researchers have used it to evaluate the effectiveness of forecasting models on
chaotic problems \cite{pathak2018model}. Predicting the solution of
a chaotic model over an interval of time which is multiple times the size of the
Lyapunov time shows the predictability of a model.

In order to find the Lyapunov time, we only need to solve for the largest LCE. 
To find this, we will use a method detailed in \cite{viswanath1998lyapunov}.

% TODO: Lyapunov calculation
% TODO: Test with Circle yielding zero

%%% DYNAMICAL SYSTEMS %%%
\section{Dynamical Systems}

Three dynamical systems will be used to demonstrate the predictability of 
chaotic systems: the Lorenz system, Poincaré's 3-body 
problem solution, and the double pendulum system.
The Poincaré system is the one of the first, and the Lorenz system
is a classic example. The double pendulum is commonly referred to as chaotic, 
and I will show this later. Both Poincaré's restricted 3 body problem solution
and 
the double pendulum are Hamiltonian systems representing physical systems
within our universe.

% TODO: numerical solutions section

\subsection{Lorenz System}

% TODO: Image of Lorenz System

The Lorenz system was originally based on a system of equations modeling 
convection created by Saltzman \cite{lorenz1963deterministic} 
\cite{saltzman1962finite}. Simplified by Lorenz, it is defined as follows
\cite{lorenz1963deterministic}:

\begin{align}
    \der{x} &= \sigma (y - x), \nonumber \\
    \der{y} &= x (\rho - z) - y, \nonumber \\
    \der{z} &= x y - \beta z. \label{eq:lorenz_equation}
\end{align}

Not all parameters for the Lorenz equation will produce chaotic behavior. In 
Lorenz's article, he uses the parameter values $\sigma=10$, $\beta=8/3$, and
$\rho=28$ \cite{lorenz1963deterministic}, and it has a maximum LCE of $0.90566$ 
\cite{viswanath1998lyapunov}. There are other parameters which
produce chaotic results outlined in Table \ref{table:lorenz_params}. Also shown 
in Table \ref{table:lorenz_params} are the values calculated by my Python
module \texttt{echonn} are listed. These values are not as accurate, but 
demonstrate the sufficient accuracy of \texttt{echonn}.

\begin{table}[H]
    \centering
    \begin{tabular}{|l|l|l|l|l|l|l|}
         % TODO: \lambda_1?
         % TODO: calculated values
        \hline
        $\sigma$ & $\rho$ & $\beta$ & actual $\lambda$ & calculated $\lambda$ & relative error of calculated $\lambda$ \\
        \hline \hline
        16 & 45.92 & 4 & 1.50255 & & \\ % DATA POINTS
        16 & 40 & 4 & 1.37446 & & \\
        10 & 28 & 8/3 & 0.90566 & & \\
        \hline
    \end{tabular}
    \caption{
        Lorenz Parameters and Largest Lyapunov Exponent
        \cite{viswanath1998lyapunov}
    }
    \label{table:lorenz_params}
\end{table}

\subsection{3-Body Problem}

The 3-body problem, or the generalized n-body problem, is a famous problem which many have 
analyzed; most notably, Herni Poincaré wrote many papers developing the area significantly
\cite{chenciner2000remarkable}. One of Poincaré's solutions, known as the circular 
restricted 3-body problem, demonstrates chaotic properties \cite{oestreicher2007history}. 
Poincaré performs several reductions of the original 3-body problem in order to come to a 
much more useful form of the problem. The mass of one of the objects is assumed to be 
negligible. This is the case for Sun-Earth-Moon orbits. The moon has little effect on the 
orbit of the sun. Another aspect is the rotating reference frame. Since the mass of one of the bodies is negligible, the problem simplifies to first solving a two body problem. While much easier to solver, Poincaré went further to stop the motion of the two planets by using a rotating reference frame. Rotating with the planets, the problem can be modeled seemingly by two static planets and a third mass-less planet orbiting around those in a two dimensional plane.

% norm from https://tex.stackexchange.com/questions/107186/how-to-write-norm-which-adjusts-its-size
\newcommand{\norm}[1]{\left\lVert#1\right\rVert}
\newcommand{\rv}{\mathbf{r}}
The N body problem is defined as follows \cite{chenciner2000remarkable}:

Let $\rv_i$ be the position of the $i^\text{th}$ body in the N body problem.
Then, the force of gravity is the sum of gravity from every other body in the system.
That is, 

\begin{equation}
   F_g^{(i)} = m_i \frac{\mathrm{d}^2 \rv_i}{\mathrm{dt}^2} =
   \sum_{j \neq i} G \frac{m_i m_j \hat{\rv}_{ij}}{\norm{\rv_{ij}}^3}
   \label{eq:nbody}
\end{equation}

where $\rv_{ij} = \rv_j - \rv_i$, $m_i$ is the mass of body $i$, and $G$ is the gravitational constant.

For our purposes, the reduced system is more useful since it is much easier to calculate than numerically solving the original 3-body problem. With three bodies in three dimension, their motion is described by 9 second order differential equations. Therefore, reducing the system to first order differential equations yields 18 first order equations. The reduced problem only requires 4 first order differential equations since the position is defined by two second order differential equations.

Derived in \cite{eberle2007case}, the system of equations defined with dimensionless rotating (synodic) coordinates:

\begin{align}
    x'' - 2 y' &= x
        - \frac{\alpha}{r_1^3} (x - \mu)
        - \frac{\mu}{r_2^3} (x + \alpha) \nonumber \\
    y'' + 2 x' &= \left(
        1 - \frac{\alpha}{r_1^3}
        - \frac{\mu}{r_2^3}
    \right) y \nonumber \\
\end{align}

where 

\begin{align}
    \mu &= \frac{m_1}{m_1 + m_2} \nonumber \\
    \alpha &= 1 - \mu \nonumber \\
    r_1(t) &= \left[(x(t) - \mu)^2 + y(t)^2\right]^{\frac{1}{2}} \nonumber \\
    r_2(t) &= \left[(x(t) + \alpha)^2 + y(t)^2\right]^{\frac{1}{2}} \label{eq:reduc_3body_cor}
\end{align}
 
and where mass body 1 and two are located at $-\alpha$ and $\mu$.

If we define the following system:

\begin{align}
    x_1 &= x \nonumber \\
    x_2 &= x' \nonumber \\
    y_1 &= y \nonumber \\
    y_2 &= y', \label{eq:first_order_def}
\end{align}

then, we can take the derivative of each:

\begin{align*}
    x_1' &= x' \\
    &= x_2,
\end{align*}
and
\begin{align*}
    x_2' &= x'' \\
    &= 2 y' + x
        - \frac{\alpha}{r_1^3} (x - \mu)
        - \frac{\mu}{r_2^3} (x + \alpha) \\
    &= 2 y_2 + x_1 - \frac{\alpha}{r_1^3} (x_1 - \mu)
        - \frac{\mu}{r_2^3} (x_1 + \alpha),
\end{align*}
and
\begin{align*}
    y_1' &= y' \\
    &= y_2,
\end{align*}
and
\begin{align*}
    y_2' &= y'' \\
    &= - 2 x' + \left(
        1 - \frac{\alpha}{r_1^3}
        - \frac{\mu}{r_2^3}
    \right) y \\
    &= - 2 x_2 + \left(
        1 - \frac{\alpha}{r_1^3}
        - \frac{\mu}{r_2^3}
    \right) y_1.
\end{align*}

Then, we can represent the second order system of motion as four first order equations:

\begin{align}
    x_1' &= x_2 \nonumber \\
    x_2' &= 2 y_2 + x_1 - \frac{\alpha}{r_1^3} (x_1 - \mu)
        - \frac{\mu}{r_2^3} (x_1 + \alpha) \nonumber \\
    y_1' &= y_2 \nonumber \\
    y_2' &= - 2 x_2 + \left(
        1 - \frac{\alpha}{r_1^3}
        - \frac{\mu}{r_2^3}
    \right) y_1, \label{eq:reduce_3_body_prog_sys} \\
\end{align}

where 

\begin{align}
    r_1 &= \left[(x_1 - \mu)^2 + y_1^2\right]^{\frac{1}{2}} \nonumber \\
    r_2 &= \left[(x_1 + \alpha)^2 + y_1^2\right]^{\frac{1}{2}}.
\end{align}

Here, $x_1$ and $y_1$ represent the position of the third body relative to the other two,
but not in the standard 
Cartesian coordinates of 3D space. Outlined in the paper \cite{eberle2007case}, the 
transformation back to Cartesian coordinates is given, but we don't need the exact 
position of the bodies since we are only concerned with the chaotic motion, so that isn't 
shown here.

There are however two issues when integrating this problem. As objects approach each
other, the distance between them 
approaches zero. If collision occurs, then this approaches infinity. There is no 
collision detection in this model, and each body has a radius of zero,
so this case is not handled. Instead, a divide by zero error occurs,
and the model cannot be calculated further.

Further, near collisions 
are also an issue. With the distance between bodies 
approaching zero, their potential energy becomes significantly smaller than other values, 
such as those describing kinetic energy \cite{chambers1999hybrid}. This 
produces a stiff system of equations, and introduces error. Since energy is conserved in
this Hamiltonian system, the error introduced represents energy being 
created. This can cause the singularity property seen in Figure \ref{fig:singularity}.
This problem can be solved by using a symplectic integrator \cite{chambers1999hybrid}. 

\begin{figure}[H]
    \centering
    \begin{subfigure}[b]{0.3\textwidth}
        \includegraphics[width=\textwidth]{images/r3b_collision.png}
        \caption{The restricted 3 body problem with collision}
        \label{fig:r3b_collision}
    \end{subfigure}
    ~
    \begin{subfigure}[b]{0.3\textwidth}
        \includegraphics[width=\textwidth]{images/r3b_collision_whole.png}
        \caption{The resultant divergence of the body}
        \label{fig:r3b_collision_whole}
    \end{subfigure}
    \caption{The Restricted 3 Body Problem Error from Collision}
    \label{fig:singularity}
\end{figure}

While it is possible to use a symplectic integrator, it is not necessary. Under the right 
initial conditions, the third body will diverge regardless. The simple divergent spiral
isn't chaotic, so instead orbital and collision less initial conditions will be used.
Without divergence and without collisions, the symplectic integrator is unnecessary.

\subsection{Double Pendulum}

The double pendulum is commonly cited to be chaotic 
\cite{stachowiak2006numerical} \cite{levien1993double}. Using the system 
of equations derived by Stachowiak \cite{stachowiak2006numerical}, the system is
defined as follows:

\begin{align}
    \der{\theta_1} &= \omega_1, \nonumber \\
    \der{\theta_2} &= \omega_2, \nonumber \\
    \der{\omega_1} &= 
    \frac{
        \sin(\theta_1 - \theta_2) \lbrack
            l_1 \cos(\theta_1 - \theta_2) \omega_1^2 + \omega_2^2
        \rbrack
    }{
        2 l_1 \lbrack
            1 + m_1 - \cos^2(\theta_1 - \theta_2)
        \rbrack
    }
    -
    \frac{
        (1 + 2 m_1) \sin \theta_1 + \sin(\theta_1 - 2 \theta_2)
    }{
        l_1 \lbrack
            1 + m_1 - \cos^2(\theta_1 - \theta_2)
        \rbrack
    }
    , \nonumber \\
    \der{\omega_2} &= \sin (\theta_1 - \theta_2) 
    \frac{
        (1+m_1) (\cos \theta_1 + l_1 \omega_1^2)
        +
        \cos(\theta_1 - \theta_2) \omega_2^2
    }{
        1 + m_1 - \cos^2(\theta_1 - \theta_2)
    }. \label{eq:doub_pen}
\end{align}

$\theta_1$ and $\omega_1$ represent the angle and angular velocity of the inner pendulum 
where $\theta_1=0$ implies the pendulum is straight down, and $\theta_1=\epsilon$ implies 
the pendulum is rotated $\epsilon$ radians counter-clockwise. The same is true for the 
outer pendulum relative to its pivot point, the end of the first pendulum. $l_1$ and $l_2$ 
are the lengths of the inner and outer pendulums, respectively. $m_1$ and $m_2$ are the 
masses of the balls on the end points of the inner and outer pendulums, respectively.
Then, the positions of the endpoints $(x_1$, $y_1)$ and $(x_2, y_2)$ of the pendulum, with 
the center pivot at the origin, $(0, 0)$, can be calculated as follows:

\begin{align}
    x_1 &= l_1 \sin(\theta_1), \nonumber \\
    y_1 &= - l_1 \cos(\theta_1), \nonumber \\
    x_2 &= x_1 + l_2 \sin(\theta_2), \nonumber \\
    y_2 &= y_1 - l_2 \cos(\theta_2), \label{eq:doub_pend_endpoints}
\end{align}

\begin{figure}[H]
    \centering
    \includegraphics[width=.5\linewidth]{images/example_doub_pend.png}
    \caption{The Double Pendulum}
    \label{fig:doub_pend}
\end{figure}

% TODO: position of the pendulum end points

While the system is a Hamiltonian system, Stachowiak decided to define the first order 
derivative components as angular velocity as opposed to momentum. This won't affect
anything regarding the purpose of this paper.

Interestingly, the 
degree of chaos is dependent on the initial conditions \cite{levien1993double}. 
In the experiment, the initial conditions had angular velocities of zero, the outer 
pendulum was straight down, and the inner pendulum varied from 0 degrees to 180 degrees 
\cite{levien1993double}; the results showed that the LCE increased with theta. 

Repeating the same experiment, but with different parameters ($l_1 = l_2 = 1$, and
$m_1 = m_2 = 1$), the results are similar (see Figure \ref{fig:doub_pend_energy}).
As inner theta increases, the largest LCE of the system seems to
approach 1.4. % DATA POINT

\begin{figure}[H]
    \centering
    \includegraphics[width=.5\linewidth]{images/chaos_vs_energy_in_doub_pend.png}
    \caption{Largest Lyapunov Characteristic Exponent vs Inner Theta Initial Condition}
    \label{fig:doub_pend_energy}
\end{figure}

%%% MODELS %%%
\section{Models}

\subsection{ARIMA}
%\subsection{Recurrent Neural Networks}
\subsection{Echo State Networks}
%\subsection{LTSM}

%%% ANALYSIS %%%
\section{Analysis}

\subsection{Method}
\subsection{Results}

%% CONCLUSION %%%
\section{Conclusion}

\bibliography{references}
\bibliographystyle{unsrt}

\end{document}

% Useful LaTeX commands

%\section{Headings: first level}
%\subsection{Headings: second level}
%\subsubsection{Headings: third level}

%\keywords{First keyword \and Second keyword \and More}

% \paragraph{Paragraph}

%% \begin{figure}
%%   \centering
%%   \fbox{\rule[-.5cm]{4cm}{4cm} \rule[-.5cm]{4cm}{0cm}}
%%   \caption{Sample figure caption.}
%%   \label{fig:fig1}
%% \end{figure}

%%\author{
  %%Lucas Wilson \\
  %%Undergraduate: Mathematics, Computer Science \\
  %%Colorado State University\\
  %%Fort Collins, CO 80523 \\
  %%\texttt{lkwilson96@gmail.com} \\
  %% \AND
  %% Coauthor \\
  %% Affiliation \\
  %% Address \\
  %% \texttt{email} \\
  %% \And
  %% Coauthor \\
  %% Affiliation \\
  %% Address \\
  %% \texttt{email} \\
  %% \And
  %% Coauthor \\
  %% Affiliation \\
  %% Address \\
  %% \texttt{email} \\
%%}
